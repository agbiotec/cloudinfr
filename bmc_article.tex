%% BioMed_Central_Tex_Template_v1.05
%%                                      %
%  bmc_article.tex            ver: 1.05 %
%                                       %


%%%%%%%%%%%%%%%%%%%%%%%%%%%%%%%%%%%%%%%%%
%%                                     %%
%%  LaTeX template for BioMed Central  %%
%%     journal article submissions     %%
%%                                     %%
%%         <27 January 2006>           %%
%%                                     %%
%%                                     %%
%% Uses:                               %%
%% cite.sty, url.sty, bmc_article.cls  %%
%% ifthen.sty. multicol.sty		       %%
%%									   %%
%%                                     %%
%%%%%%%%%%%%%%%%%%%%%%%%%%%%%%%%%%%%%%%%%


%%%%%%%%%%%%%%%%%%%%%%%%%%%%%%%%%%%%%%%%%%%%%%%%%%%%%%%%%%%%%%%%%%%%%
%%                                                                 %%	
%% For instructions on how to fill out this Tex template           %%
%% document please refer to Readme.pdf and the instructions for    %%
%% authors page on the biomed central website                      %%
%% http://www.biomedcentral.com/info/authors/                      %%
%%                                                                 %%
%% Please do not use \input{...} to include other tex files.       %%
%% Submit your LaTeX manuscript as one .tex document.              %%
%%                                                                 %%
%% All additional figures and files should be attached             %%
%% separately and not embedded in the \TeX\ document itself.       %%
%%                                                                 %%
%% BioMed Central currently use the MikTex distribution of         %%
%% TeX for Windows) of TeX and LaTeX.  This is available from      %%
%% http://www.miktex.org                                           %%
%%                                                                 %%
%%%%%%%%%%%%%%%%%%%%%%%%%%%%%%%%%%%%%%%%%%%%%%%%%%%%%%%%%%%%%%%%%%%%%


\NeedsTeXFormat{LaTeX2e}[1995/12/01]
\documentclass[10pt]{bmc_article}    



% Load packages
\usepackage{cite} % Make references as [1-4], not [1,2,3,4]
\usepackage{url}  % Formatting web addresses  
\usepackage{ifthen}  % Conditional 
\usepackage{multicol}   %Columns
\usepackage[utf8]{inputenc} %unicode support
%\usepackage[applemac]{inputenc} %applemac support if unicode package fails
%\usepackage[latin1]{inputenc} %UNIX support if unicode package fails
\urlstyle{rm}
 
 
%%%%%%%%%%%%%%%%%%%%%%%%%%%%%%%%%%%%%%%%%%%%%%%%%	
%%                                             %%
%%  If you wish to display your graphics for   %%
%%  your own use using includegraphic or       %%
%%  includegraphics, then comment out the      %%
%%  following two lines of code.               %%   
%%  NB: These line *must* be included when     %%
%%  submitting to BMC.                         %% 
%%  All figure files must be submitted as      %%
%%  separate graphics through the BMC          %%
%%  submission process, not included in the    %% 
%%  submitted article.                         %% 
%%                                             %%
%%%%%%%%%%%%%%%%%%%%%%%%%%%%%%%%%%%%%%%%%%%%%%%%%                     


\def\includegraphic{}
\def\includegraphics{}



\setlength{\topmargin}{0.0cm}
\setlength{\textheight}{21.5cm}
\setlength{\oddsidemargin}{0cm} 
\setlength{\textwidth}{16.5cm}
\setlength{\columnsep}{0.6cm}

\newboolean{publ}

%%%%%%%%%%%%%%%%%%%%%%%%%%%%%%%%%%%%%%%%%%%%%%%%%%
%%                                              %%
%% You may change the following style settings  %%
%% Should you wish to format your article       %%
%% in a publication style for printing out and  %%
%% sharing with colleagues, but ensure that     %%
%% before submitting to BMC that the style is   %%
%% returned to the Review style setting.        %%
%%                                              %%
%%%%%%%%%%%%%%%%%%%%%%%%%%%%%%%%%%%%%%%%%%%%%%%%%%
 

%Review style settings
\newenvironment{bmcformat}{\begin{raggedright}\baselineskip20pt\sloppy\setboolean{publ}{false}}{\end{raggedright}\baselineskip20pt\sloppy}

%Publication style settings
%\newenvironment{bmcformat}{\fussy\setboolean{publ}{true}}{\fussy}



% Begin ...
\begin{document}
\begin{bmcformat}


%%%%%%%%%%%%%%%%%%%%%%%%%%%%%%%%%%%%%%%%%%%%%%
%%                                          %%
%% Enter the title of your article here     %%
%%                                          %%
%%%%%%%%%%%%%%%%%%%%%%%%%%%%%%%%%%%%%%%%%%%%%%

\title{Eight Questions and Answers Defining Cloud Computing for the Digital Biology Era}
 
%%%%%%%%%%%%%%%%%%%%%%%%%%%%%%%%%%%%%%%%%%%%%%
%%                                          %%
%% Enter the authors here                   %%
%%                                          %%
%% Ensure \and is entered between all but   %%
%% the last two authors. This will be       %%
%% replaced by a comma in the final article %%
%%                                          %%
%% Ensure there are no trailing spaces at   %% 
%% the ends of the lines                    %%     	
%%                                          %%
%%%%%%%%%%%%%%%%%%%%%%%%%%%%%%%%%%%%%%%%%%%%%%


\author{Konstantinos Krampis\correspondingauthor$^{1}$%
       \email{Konstantinos Krampis\correspondingauthor - agbiotec@gmail.com}%
      }
      

%%%%%%%%%%%%%%%%%%%%%%%%%%%%%%%%%%%%%%%%%%%%%%
%%                                          %%
%% Enter the authors' addresses here        %%
%%                                          %%
%%%%%%%%%%%%%%%%%%%%%%%%%%%%%%%%%%%%%%%%%%%%%%

\address{%
    \iid(1)Informatics Department, J. Craig Venter Institute, 9704 Medical Center Dr. ,%
        Rockville, MD 20850, USA
}%

\maketitle

%%%%%%%%%%%%%%%%%%%%%%%%%%%%%%%%%%%%%%%%%%%%%%
%%                                          %%
%% The Abstract begins here                 %%
%%                                          %%
%% The Section headings here are those for  %%
%% a Research article submitted to a        %%
%% BMC-Series journal.                      %%  
%%                                          %%
%% If your article is not of this type,     %%
%% then refer to the Instructions for       %%
%% authors on http://www.biomedcentral.com  %%
%% and change the section headings          %%
%% accordingly.                             %%   
%%                                          %%
%%%%%%%%%%%%%%%%%%%%%%%%%%%%%%%%%%%%%%%%%%%%%%


\begin{abstract}
        % Do not use inserted blank lines (ie \\) until main body of text.
        \paragraph*{Background:} Text for this section of the abstract. \cite{Metzker2009} 
      
        \paragraph*{Results:} Text for this section of the abstract \ldots

        \paragraph*{Conclusions:} Text for this section of the abstract \ldots
\end{abstract}



\ifthenelse{\boolean{publ}}{\begin{multicols}{2}}{}




%%%%%%%%%%%%%%%%%%%%%%%%%%%%%%%%%%%%%%%%%%%%%%
%%                                          %%
%% The Main Body begins here                %%
%%                                          %%
%% The Section headings here are those for  %%
%% a Research article submitted to a        %%
%% BMC-Series journal.                      %%  
%%                                          %%
%% If your article is not of this type,     %%
%% then refer to the instructions for       %%
%% authors on:                              %%
%% http://www.biomedcentral.com/info/authors%%
%% and change the section headings          %%
%% accordingly.                             %% 
%%                                          %%
%% See the Results and Discussion section   %%
%% for details on how to create sub-sections%%
%%                                          %%
%% use \cite{...} to cite references        %%
%%  \cite{koon} and                         %%
%%  \cite{oreg,khar,zvai,xjon,schn,pond}    %%
%%  \nocite{smith,marg,hunn,advi,koha,mouse}%%
%%                                          %%
%%%%%%%%%%%%%%%%%%%%%%%%%%%%%%%%%%%%%%%%%%%%%%




%%%%%%%%%%%%%%%%
%% Background %%
%%
\section*{Background: Sequencing Technologies and The New Era of Digital Biology}

      Digital Biology Research Enabled by High-Throughput Sequencing is defined by... \pb
             
      Sequencing technologies continue to move in a direction where throughput per run is
      increasing while the cost per basepair is decreasing (review in \cite{Mason2012}). 
      Several technologies available on the market today produce massive volumes of sequence data per
      run; for instance, one of the most widely used instruments in the field for the past few years
      and currently, Illumina's GAIIx system can produce up to 95 Giga-base of sequence (Gb) per run 
      \cite{Illumina} while the also broadly used SOLiD sequencer has yields of a similar range 
      up to 90 Gb \cite{solid5500}. With the latest generation of instruments including Illumina’s 
      HiSeq systems the yield per run has reached 600 Gb \cite{Illumina}, and with the the Pacific 
      BioSciences instrument yields of 90 Gb can be achieved \cite{}. \pb
      
      Recently, small-factor, benchtop sequencers became available including GS Junior by 454, 
      MiSeq by Illumina and Ion Proton by Life Technologies, all of which can be acquired at a fraction 
      of the cost and be affordable for independent researchers running smaller laboratories. Nonetheless 
      these sequencers still provide enough capacity at 0.035Gb, 1Gb and 1.5Gb respectively for GS Junior, 
      Ion Proton and MiSeq review in \cite{Loman2012}) for sequencing bacterial, small fungal or viral genomes. 
      This fact in combination of the low cost per run (US \$225 -\$1100 depending on which benchtop 
      sequencer is used and required throughput), can establish sequencing as standard technique for
      basic biological research. Examples include Single Nucleotide Polymorphism (SNP) 
      variation discovery , gene expression analysis (RNAseq), DNA–protein interaction analysis 
      (ChiPseq), (review in \cite{Mardis2008}).
The new generation of sequencing technologies is also being 
      used in the area of metagenomics, for large-scale studies of uncultivated microbial communities. 
      The J. Craig Venter Institute (JCVI) for example has been involved in several such metagenomic 
      projects, including the Sorcerer II Global Ocean Sampling (GOS, \cite{Nelson2010}) expedition to study marine 
      microbial diversity, and also the National Institutes of Health funded Human Microbiome Project 
      to study human associated microbial communities \cite{Rusch2007}. \pb
      

 
%%%%%%%%%%%%%%%%%%%%%%%%%%%%
%% Results and Discussion %%
%%
\section*{Discusion: Eight Questions and Answers to Define Cloud Computing Applications on Digital Biology}\pb 

  \subsection*{What is the Role of Cloud Computing Play in The Digital Biology Research Era?}
  
      While large datasets are generated during sequencing runs, sequencers are typically bundled 
      with only minimal computational and storage capacity for data capture during the run. 
      For example, the un-assembled reads returned from a single lane of the Illumina GAIIx instrument 
      after base calling are approximately 100 GigaByte (GB) in size. Given the scale of datasets, 
      scientific value cannot necessarily be obtained from the investment in a sequencing instrument,
      unless it is accompanied by an equal investment in a large-scale bioinformatics infrastructure. \pb

      For small laboratories acquiring a sequencing instrument, the currently available online
      software tools are not and option for downstream sequence analysis, since they cannot
      provide the required compute capacity. The NCBI website for example (Johnson et al. 2008),
      cannot accept input sequence data files of 0.5 GB size for BLAST sequence similarity
      search. With this as an example, we see that scientific value cannot be obtained from an
      investment in sequencing instruments, unless it is accompanied by an almost equal or
      greater investment in informatics hardware infrastructure. Besides large capacity compute
      servers, also required are trained bioinformaticians competent to install, configure and
      use specific software to analyze the generated data, and store the data in appropriate
      formats for future use (Richer et al. 2009).
\pb

  \subsection*{Renting Cloud Computers Versus Building Local Clusters?}

      Even if bioinformatics computing clusters are accessible by researchers, a problem
      is related to the sub-optimal utilization of the hardware, and its associated
      maintenance costs. This is due to the nature of bioinformatics for next-generation
      sequencing, where computationally intensive tasks of genome assembly or whole genome
      alignments require extensive compute resources, while tasks such as genome
      annotation and browsing are less computationally demanding. For smaller laboratories
      this can become a hurdle, as in addition to getting funds for building a cluster
      with capacity to handle the large computations, they need to come up with the money
      for maintaining a system that is not utilized at its full capacity most of the time.
      A second problem arises from the fact that  the public sequence databases are
      constantly growing in size. These databases must be downloaded to local storage
      systems, in order to be used for example when conducting comparative genome
      annotations. As these databases get larger the process becomes more time consuming,
      incurring in higher bandwidth and storage costs for replicating the data locally.
      Finally, building a bioinformatics infrastructure for next-generation sequencing
      also involves hiring trained bioinformaticians competent to install, configure and
      use specialized software tools and data analysis pipelines, which can present a
      higher expense than that of acquiring the computing cluster.

      ur system will not be limited only to the Amazon EC2 cloud platform, since researchers
      that have access to a local cluster at their home institution will have the option to
      download the VM and run OSMF, without being required to perform any software installation.
      The only dependency will be a virtualization layer that can run the VM on the cluster such
      as for example the OpenStack open-source cloud (http://www.openstack.org). OpenStack is
      available as part of widely used Ubuntu Linux (http://www.ubuntu.com/cloud) and included
      by default on a compute cluster set-up to run this operating system, while it can be
      easily installed as a package on clusters running other Linux versions. Alternatively, for
      researchers that do not have access to local compute clouds, OpenStack installations can
      be accessed through the government-funded Argonne National Lab Magellan Cloud
      http://www.alcf.anl.gov/magellan) that provides compute allocations to researchers, in
      addition to a number of academic computing centers in both the US and abroad
      (http://openstack.org/user-stories/), or commercial cloud providers such as RackSpace
      (http://www.rackspace.com/cloud/private_edition/openstack/)


  \subsection*{Are Cloud-Based Bioinformatics Software Suites Available on the Cloud ?}

      For the sequence analysis suites currently available on the cloud, the common pattern is
      packaging existing bioinformatics applications within virtual compute servers. This is a
      great solution for smaller labs that lack informatics infrastructure, since it makes
      available pre-configured software and on-demand computing using virtualized
      infrastructures. However, a problem exists with this approach in regards to the monolithic
      design of existing bioinformatics applications, which are ported to the cloud. These
      applications usually process data serially, and are not designed to leverage specific
      characteristics of cloud computing platforms, such as highly parallelism and distributed
      computing. Therefore, they cannot scale efficiently as the amount of data increases
      (Wilkening et al. 2009). According to the same authors, it is difficult to transfer on a
      cloud infrastructure specialized bioinformatics applications such as the MG-RAST
      metagenomics analysis pipeline for example (Glass et al. 2010). Despite being designed to
      scale in parallel for large sequences datasets, MG-RAST's design makes it tightly coupled
      with specific cluster computing hardware and the SunGrid Engine (online ref.10) scheduling
      framework.

  \subsection*{Can Non-Computationally Savvy Researchers Easily Access the Cloud?}
    Text for this sub-section.  More results \ldots
  
  \subsection*{What the Public, Private, Open-Source or Commercial Cloud Available to Biologists Today ?}
    Text for this sub-section.  More results \ldots
  
  \subsection*{Available Options to Buying Software as A Service on the Cloud ?}

      During the past year, both public and commercial offerings of pre-configured sequence
      analysis applications on the cloud have become available. On the commercial side, DNAnexus
      (online ref. 3) currently includes tools for ChiPseq, RNAseq, 3'-end sequencing for
      expression quantification (3SEQ) and enzyme restriction analysis. DNAnexus runs on the
      Amazon Elastic Compute Cloud (EC2, online ref. 4), which provides on-demand virtual
      servers with various compute capacities. Another offering is iNquiry, which is a port to
      the cloud of the bioinformatics software suite that used to be bundled together with the
      computer clusters built by the BioTeam (online ref. 5), but is no longer maintained. This
      platform is essentially a web-server for pre-installed open-source tools such as EMBOSS,
      HMMER, BLAST and the R statistical package, also on the EC2 cloud platform.

In regards to
      public offerings, the Galaxy bioinformatics workbench (online ref. 6), includes a range of
      software, from scripts that extract entries from FASTA files, to tools for processing
      next-generation sequence data. It is a self-contained platform including a web server
      along with the bioinformatics tools, and users can easily add more software packages by
      editing simple configuration scripts. Galaxy has recently been ported to run on the Amazon
      EC2 compute cloud. Another public offering for computing on the cloud has come through our
      own work on JCVI Cloud Biolinux (online ref. 7), which is a virtual high performance
      computing server on Amazon EC2. Our offering bundles a set of sequence analysis tools
      similar to those offered by iNquiry, with the difference being that the virtual server is
      available for download (online ref. 8), and users can run it on the open-source Eucalyptus
      (online ref. 9) or Science Clouds platforms (Keahey et al. 2009). In addition, Cloud
      Biolinux also includes the Celera genome assembler and a set of scripts that allow for
      push-button creation of virtual computing clusters for parallel BLAST, geared towards
      researchers that intend to perform large scale genomic sequence analysis.

For the
      sequence analysis suites currently available on the cloud, the common pattern is packaging
      existing bioinformatics applications within virtual compute servers. This is a great
      solution for smaller labs that lack informatics infrastructure, since it makes available
      pre-configured software and on-demand computing using virtualized infrastructures.
      However, a problem exists with this approach in regards to the monolithic design of
      existing bioinformatics applications, which are ported to the cloud. These applications
      usually process data serially, and are not designed to leverage specific
      characteristics of cloud computing platforms, such as highly parallelism and distributed
      computing. Therefore, they cannot scale efficiently as the amount of data increases
      (Wilkening et al. 2009). According to the same authors, it is difficult to transfer on a
      cloud infrastructure specialized bioinformatics applications such as the MG-RAST
      metagenomics analysis pipeline for example (Glass et al. 2010). Despite being designed to
      scale in parallel for large sequences datasets, MG-RAST's design makes it tightly coupled
      with specific cluster computing hardware and the SunGrid Engine (online ref.10) scheduling
      framework.

  \subsection*{Which Factors Challenge Adoption of Cloud-Based Solutions for Bioinformatics }

      Another important concern for cloud-based bioinformatic tools is related to the data
      transfer bottleneck from the local sequencing machines to the cloud servers. According to
      the data published for the Amazon cloud platform (online ref.11), 600GB of data would
      require approximately one week to upload on to the remote cloud servers, when using an
      average broadband connection of 10Mbps. With a faster T3 connection which is usually
      easily obtainable even at small research institutions, within one week 2TB of data can be
      uploaded or approximately 600GB in 2 days.  Solutions addressing this issue are available
      both as software that maximizes data transfer over the network compared to traditional
      File Transfer Protocol (FTP), or physical disk drive import/export services offered by the
      cloud provider to its customers. Aspera's server (online ref. 12) has been recently
      integrated to NCBI's infrastructure, and researchers can download a free client that
      allows increased upload speeds to the Short Read Archive (online ref. 13).  Through the
      Aspera software, transfer bandwidth between NCBI and the European Bioinformatics Institute
      for data sharing in the 1000 Genomes Project, has been increased from 20Mbps to 1000Mbps
      (see online ref. 14). 

Finally, the Amazon offers the option for its users to physically
      ship disk drives to the company's offices and have the data copied to their servers
      (online ref. 11). With only 80 import cost for disk drives up to 4TB of data (4000GB),
      this is the most efficient method if we take into account the charge by Amazon for 0.10
      per GB of bandwith consumed, which would add up to 60 for a 600GB data upload. In
      addition to that cost, the expense for obtaining a high-bandwidth internet connection for
      the data upload should be taken into account. We expect the Microsoft Azure cloud platform
      to offer a similar service in the near future, given the requests on the Azure developer
      forums and the immediate consideration of the matter by Microsoft (online ref. 15).
  
  \subsection*{Yet another results sub-heading}
    Text for this sub-section.  More results \ldots


%%%%%%%%%%%%%%%%%%%%%%
\section*{Conclusions}
  Text for this section \ldots


  
%%%%%%%%%%%%%%%%%%
\section*{Methods}
  \subsection*{Methods sub-heading for this section}
    Text for this sub-section \ldots

  \subsection*{Another methods sub-heading for this section}
    Text for this sub-section \ldots

  \subsection*{Yet another sub-heading for this section}
    Text for this sub-section \ldots


    
%%%%%%%%%%%%%%%%%%%%%%%%%%%%%%%%
\section*{Authors contributions}
    Text for this section \ldots

    

%%%%%%%%%%%%%%%%%%%%%%%%%%%
\section*{Acknowledgements}
  \ifthenelse{\boolean{publ}}{\small}{}
  Text for this section \ldots


 
%%%%%%%%%%%%%%%%%%%%%%%%%%%%%%%%%%%%%%%%%%%%%%%%%%%%%%%%%%%%%
%%                  The Bibliography                       %%
%%                                                         %%              
%%  Bmc_article.bst  will be used to                       %%
%%  create a .BBL file for submission, which includes      %%
%%  XML structured for BMC.                                %%
%%                                                         %%
%%                                                         %%
%%  Note that the displayed Bibliography will not          %% 
%%  necessarily be rendered by Latex exactly as specified  %%
%%  in the online Instructions for Authors.                %% 
%%                                                         %%
%%%%%%%%%%%%%%%%%%%%%%%%%%%%%%%%%%%%%%%%%%%%%%%%%%%%%%%%%%%%%


{\ifthenelse{\boolean{publ}}{\footnotesize}{\small}
 \bibliographystyle{bmc_article}  % Style BST file
  \bibliography{bmc_article} }     % Bibliography file (usually '*.bib' ) 

%%%%%%%%%%%

\ifthenelse{\boolean{publ}}{\end{multicols}}{}

%%%%%%%%%%%%%%%%%%%%%%%%%%%%%%%%%%%
%%                               %%
%% Figures                       %%
%%                               %%
%% NB: this is for captions and  %%
%% Titles. All graphics must be  %%
%% submitted separately and NOT  %%
%% included in the Tex document  %%
%%                               %%
%%%%%%%%%%%%%%%%%%%%%%%%%%%%%%%%%%%

%%
%% Do not use \listoffigures as most will included as separate files

\section*{Figures}
  \subsection*{Figure 1 - Sample figure title}
      A short description of the figure content
      should go here.

  \subsection*{Figure 2 - Sample figure title}
      Figure legend text.



%%%%%%%%%%%%%%%%%%%%%%%%%%%%%%%%%%%
%%                               %%
%% Tables                        %%
%%                               %%
%%%%%%%%%%%%%%%%%%%%%%%%%%%%%%%%%%%

%% Use of \listoftables is discouraged.
%%
\section*{Tables}
  \subsection*{Table 1 - Sample table title}
    Here is an example of a \emph{small} table in \LaTeX\ using  
    \verb|\tabular{...}|. This is where the description of the table 
    should go. \par \mbox{}
    \par
    \mbox{
      \begin{tabular}{|c|c|c|}
        \hline \multicolumn{3}{|c|}{My Table}\\ \hline
        A1 & B2  & C3 \\ \hline
        A2 & ... & .. \\ \hline
        A3 & ..  & .  \\ \hline
      \end{tabular}
      }
  \subsection*{Table 2 - Sample table title}
    Large tables are attached as separate files but should
    still be described here.



%%%%%%%%%%%%%%%%%%%%%%%%%%%%%%%%%%%
%%                               %%
%% Additional Files              %%
%%                               %%
%%%%%%%%%%%%%%%%%%%%%%%%%%%%%%%%%%%

\section*{Additional Files}
  \subsection*{Additional file 1 --- Sample additional file title}
    Additional file descriptions text (including details of how to
    view the file, if it is in a non-standard format or the file extension).  This might
    refer to a multi-page table or a figure.

  \subsection*{Additional file 2 --- Sample additional file title}
    Additional file descriptions text.


\end{bmcformat}
\end{document}







