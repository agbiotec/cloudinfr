%% BioMed_Central_Tex_Template_v1.05
%%                                      %
%  bmc_article.tex            ver: 1.05 % %


%%%%%%%%%%%%%%%%%%%%%%%%%%%%%%%%%%%%%%%%%
%%                                     %%
%%  LaTeX template for BioMed Central  %%
%%     journal article submissions     %%
%%                                     %%
%%         <27 January 2006>           %%
%%                                     %%
%%                                     %%
%% Uses:                               %%
%% cite.sty, url.sty, bmc_article.cls  %%
%% ifthen.sty. multicol.sty		       %%
%%									   %%
%%                                     %%
%%%%%%%%%%%%%%%%%%%%%%%%%%%%%%%%%%%%%%%%%


%%%%%%%%%%%%%%%%%%%%%%%%%%%%%%%%%%%%%%%%%%%%%%%%%%%%%%%%%%%%%%%%%%%%%
%%                                                                 %%	
%% For instructions on how to fill out this Tex template           %%
%% document please refer to Readme.pdf and the instructions for    %%
%% authors page on the biomed central website                      %%
%% http://www.biomedcentral.com/info/authors/                      %%
%%                                                                 %%
%% Please do not use \input{...} to include other tex files.       %%
%% Submit your LaTeX manuscript as one .tex document.              %%
%%                                                                 %%
%% All additional figures and files should be attached             %%
%% separately and not embedded in the \TeX\ document itself.       %%
%%                                                                 %%
%% BioMed Central currently use the MikTex distribution of         %%
%% TeX for Windows) of TeX and LaTeX.  This is available from      %%
%% http://www.miktex.org                                           %%
%%                                                                 %%
%%%%%%%%%%%%%%%%%%%%%%%%%%%%%%%%%%%%%%%%%%%%%%%%%%%%%%%%%%%%%%%%%%%%%


\NeedsTeXFormat{LaTeX2e}[1995/12/01] \documentclass[10pt]{bmc_article}    



% Load packages
\usepackage{cite} % Make references as [1-4], not [1,2,3,4] \usepackage{url}  % Formatting web addresses
\usepackage{ifthen}  % Conditional \usepackage{multicol}   %Columns \usepackage[utf8]{inputenc} %unicode support
%\usepackage[applemac]{inputenc} %applemac support if unicode package fails \usepackage[latin1]{inputenc}
%%UNIX support if unicode package fails \urlstyle{rm}


%%%%%%%%%%%%%%%%%%%%%%%%%%%%%%%%%%%%%%%%%%%%%%%%%	
%%                                             %%
%%  If you wish to display your graphics for   %%
%%  your own use using includegraphic or       %%
%%  includegraphics, then comment out the      %%
%%  following two lines of code.               %%   
%%  NB: These line *must* be included when     %%
%%  submitting to BMC.                         %% 
%%  All figure files must be submitted as      %%
%%  separate graphics through the BMC          %%
%%  submission process, not included in the    %% 
%%  submitted article.                         %% 
%%                                             %%
%%%%%%%%%%%%%%%%%%%%%%%%%%%%%%%%%%%%%%%%%%%%%%%%%                     


\def\includegraphic{} \def\includegraphics{}



\setlength{\topmargin}{0.0cm} \setlength{\textheight}{21.5cm} \setlength{\oddsidemargin}{0cm}
\setlength{\textwidth}{16.5cm} \setlength{\columnsep}{0.6cm}

\newboolean{publ}

%%%%%%%%%%%%%%%%%%%%%%%%%%%%%%%%%%%%%%%%%%%%%%%%%%
%%                                              %%
%% You may change the following style settings  %%
%% Should you wish to format your article       %%
%% in a publication style for printing out and  %%
%% sharing with colleagues, but ensure that     %%
%% before submitting to BMC that the style is   %%
%% returned to the Review style setting.        %%
%%                                              %%
%%%%%%%%%%%%%%%%%%%%%%%%%%%%%%%%%%%%%%%%%%%%%%%%%%


%Review style settings
\newenvironment{bmcformat}{\begin{raggedright}\baselineskip20pt\sloppy\setboolean{publ}{false}}{\end{raggedright}\baselineskip20pt\sloppy}

%Publication style settings \newenvironment{bmcformat}{\fussy\setboolean{publ}{true}}{\fussy}



% Begin ...
\begin{document} \begin{bmcformat}


%%%%%%%%%%%%%%%%%%%%%%%%%%%%%%%%%%%%%%%%%%%%%%
%%                                          %%
%% Enter the title of your article here     %%
%%                                          %%
%%%%%%%%%%%%%%%%%%%%%%%%%%%%%%%%%%%%%%%%%%%%%%

\title{Eight Questions and Answers Defining Cloud Computing for the Digital Biology Era}

%%%%%%%%%%%%%%%%%%%%%%%%%%%%%%%%%%%%%%%%%%%%%%
%%                                          %%
%% Enter the authors here                   %%
%%                                          %%
%% Ensure \and is entered between all but   %%
%% the last two authors. This will be       %%
%% replaced by a comma in the final article %%
%%                                          %%
%% Ensure there are no trailing spaces at   %% 
%% the ends of the lines                    %%     	
%%                                          %%
%%%%%%%%%%%%%%%%%%%%%%%%%%%%%%%%%%%%%%%%%%%%%%


\author{Konstantinos Krampis\correspondingauthor$^{1}$% \email{Konstantinos Krampis\correspondingauthor -
agbiotec@gmail.com}% }


%%%%%%%%%%%%%%%%%%%%%%%%%%%%%%%%%%%%%%%%%%%%%%
%%                                          %%
%% Enter the authors' addresses here        %%
%%                                          %%
%%%%%%%%%%%%%%%%%%%%%%%%%%%%%%%%%%%%%%%%%%%%%%

\address{% \iid(1)Informatics Department, J. Craig Venter Institute, 9704 Medical Center Dr. ,% Rockville, MD
20850, USA }%

\maketitle

%%%%%%%%%%%%%%%%%%%%%%%%%%%%%%%%%%%%%%%%%%%%%%
%%                                          %%
%% The Abstract begins here                 %%
%%                                          %%
%% The Section headings here are those for  %%
%% a Research article submitted to a        %%
%% BMC-Series journal.                      %%  
%%                                          %%
%% If your article is not of this type,     %%
%% then refer to the Instructions for       %%
%% authors on http://www.biomedcentral.com  %%
%% and change the section headings          %%
%% accordingly.                             %%   
%%                                          %%
%%%%%%%%%%%%%%%%%%%%%%%%%%%%%%%%%%%%%%%%%%%%%%


\begin{abstract}
% Do not use inserted blank lines (ie \\) until main body of text.
\paragraph*{Background:} Text for this section of the abstract. \cite{Metzker2009} 

\paragraph*{Results:} Text for this section of the abstract \ldots

\paragraph*{Conclusions:} Text for this section of the abstract \ldots \end{abstract}



\ifthenelse{\boolean{publ}}{\begin{multicols}{2}}{}




%%%%%%%%%%%%%%%%%%%%%%%%%%%%%%%%%%%%%%%%%%%%%%
%%                                          %%
%% The Main Body begins here                %%
%%                                          %%
%% The Section headings here are those for  %%
%% a Research article submitted to a        %%
%% BMC-Series journal.                      %%  
%%                                          %%
%% If your article is not of this type,     %%
%% then refer to the instructions for       %%
%% authors on:                              %%
%% http://www.biomedcentral.com/info/authors%%
%% and change the section headings          %%
%% accordingly.                             %% 
%%                                          %%
%% See the Results and Discussion section   %%
%% for details on how to create sub-sections%%
%%                                          %%
%% use \cite{...} to cite references        %%
%%  \cite{koon} and                         %%
%%  \cite{oreg,khar,zvai,xjon,schn,pond}    %%
%%  \nocite{smith,marg,hunn,advi,koha,mouse}%%
%%                                          %%
%%%%%%%%%%%%%%%%%%%%%%%%%%%%%%%%%%%%%%%%%%%%%%




%%%%%%%%%%%%%%%%
%% Background %%
%%
\section*{Background: Sequencing Technologies and The New Era of Digital Biology}

Digital Biology Research Enabled by High-Throughput Sequencing is defined by... \pb

Sequencing technologies continue to move in a direction where throughput per run is increasing while the cost
per basepair is decreasing (review in \cite{Mason2012}).  Several technologies available on the market today
produce massive volumes of sequence data per run; for instance, one of the most widely used instruments in the
field for the past few years and currently, Illumina's GAIIx system can produce up to 95 Giga-base of sequence
(Gb) per run \cite{Illumina} while the also broadly used SOLiD sequencer has yields of a similar range up to
90 Gb \cite{solid5500}. With the latest generation of instruments including Illumina’s HiSeq systems the yield
per run has reached 600 Gb \cite{Illumina}, and with the the Pacific BioSciences instrument yields of 90 Gb
can be achieved \cite{}. \pb

Recently, small-factor, benchtop sequencers became available including GS Junior by 454, MiSeq by Illumina and
Ion Proton by Life Technologies, all of which can be acquired at a fraction of the cost and be affordable for
independent researchers running smaller laboratories. Nonetheless these sequencers still provide enough
capacity at 0.035Gb, 1Gb and 1.5Gb respectively for GS Junior, Ion Proton and MiSeq review in
\cite{Loman2012}) for sequencing bacterial, small fungal or viral genomes.  This fact in combination of the
low cost per run (US \$225 -\$1100 depending on which benchtop sequencer is used and required throughput), can
establish sequencing as standard technique for basic biological research.  Examples include Single Nucleotide
Polymorphism (SNP) variation discovery , gene expression analysis (RNAseq), DNA–protein interaction analysis
(ChiPseq), (review in \cite{Mardis2008}).

The new generation of sequencing technologies is also being used in the area of metagenomics, for large-scale
studies of uncultivated microbial communities.  The J.  Craig Venter Institute (JCVI) for example has been
involved in several such metagenomic projects, including the Sorcerer II Global Ocean Sampling (GOS,
\cite{Rusch2007}) expedition to study marine microbial diversity, and also the National Institutes of Health
funded Human Microbiome Project to study human associated microbial communities \cite{Nelson2010}. \pb



%%%%%%%%%%%%%%%%%%%%%%%%%%%%
%% Results and Discussion %%
%%
\section*{Discusion: Eight Questions and Answers to Define Cloud Computing Applications on Digital Biology} \pb 

\subsection*{What is the Role Cloud Computing Plays in The Digital Biology Research Era?}

While large datasets are generated during sequencing runs, sequencers are typically bundled with only minimal
computational and storage capacity for data capture during the run.  For example, the un-assembled reads
returned from a single lane of the Illumina GAIIx instrument after base calling are approximately 100 GigaByte
(GB) in size. Given the scale of datasets, scientific value cannot necessarily be obtained from the investment
in a sequencing instrument, unless it is accompanied by an equal investment in a large-scale bioinformatics
infrastructure. \pb

For small laboratories acquiring a sequencing instrument, the currently available online software tools are
not and option for downstream sequence analysis, since they cannot provide the required compute capacity. The
NCBI website for example \cite{johnson2008ncbi}, cannot accept input sequence data files of more than 0.5 GB
size for BLAST sequence similarity search. With this as an example, we see that scientific value cannot be
obtained from an investment in sequencing instruments, unless it is accompanied by an almost equal or greater
investment in informatics hardware infrastructure. Besides large capacity compute servers, also required are
trained bioinformaticians competent to install, configure and use specific software to analyze the generated
data, and store the data in appropriate formats for future use.  \pb

\subsection*{Renting Cloud Computers Versus Building Local Clusters?}

Due to the nature of next-generation sequencing data analysis, computationally intensive tasks of genome
assembly or whole genome alignments requiring extensive compute resources, alternate with tasks such as genome
annotation and curation that are less computationally demanding. This leads to  sub-optimal utilization of
computer hardware installed in server clusters within  data centers, while maintenance costs stay at constant
levels. For smaller academic institution this can pose a significant barrier of entry to leveraging genomic
sequencing for research, as in addition to getting funds for building a cluster with capacity to handle the
large , they also need to maintain  a system that is not utilized at its full capacity most of the time.  A
second problem arises from the fact that the public sequence databases are constantly growing in size, and in
most types of bioinformatic analysis they need to be downloaded local storage systems, in order to be used as
reference for example when conducting comparative genome annotations. As these databases get larger the
process becomes more time consuming, incurring in higher bandwidth and storage costs for replicating the data
locally.  Finally, building a data analysis infrastructure for next-generation sequencing also involves hiring
trained bioinformatics engineers competent to install, configure and use specialized software tools and data
analysis pipelines, which can present a higher expense than that of acquiring computer hardware. \pb


\subsection*{Are Cloud-Based Bioinformatics Software Suites Available on the Cloud ?}
  
A number of systems for  deploying bioinformatics tools through web portals have been
developed during the past decade, including the Biology Workbench \cite{Subramaniam1998Biology}
, PISE \cite{Letondal2001Web}, wEMBOSS \cite{Sarachu2005WEMBOSS}, Mobyle \cite{Neron2009Mobyle}
BioManager \cite{Cattley2007BioManager} and BioExtract \cite{Lushbough2008Implementing}. Some 
of those systems are not actively developed anymore; most require significant software development effort and system
customization in order to deploy the tools through a web interface; with the exception of Mobyle, none of
these portals provides users with the option to create and edit complex data analysis workflows; while these
portals are accessible online, they do not leverage the Cloud's scalability but rather run on physical servers
with fixed computational capacity, presenting a limitation on the scale of data analysis that can be
performed;  users have the option to download the source code for each system, but need to provision the 
computer hardware and software engineering expertise to install the portal on a local server; finally, management 
and sharing of datasets among users is difficult, while available storage space is limited by the capacity of the server
where the portal is installed.

On the other hand, web portals that can handle large-scale datasets, have been developed by well-funded
big institutions including IMG/M \cite{Grigoriev2012}, CAMERA \cite{Altintas2010}, EBI \cite{Hunter2011} 
and MG-RAST \cite{Aziz2010}. These portals offer considerable compute resources and data storage for 
large-scale data analysis on compute clusters, but a major drawback for example for IMG/M is that the 
software is not open-source. These centralized services cannot possibly support the numerous 
small labs that acquire sequence data but are limited by the availability of computational resources or 
the requisite informatics expertise, while they all require researchers to go through an application and
approval process for getting access to the resource. Furthemore, often datasets are redundantly submitted 
to all centers since each has its own annotation pipeline and a slightly different set of tools.  

Alternative to these centralized services, Cloud-based, scalable data analysis portals such as Galaxy
\cite{Goecks2010}, CloVR \cite{Angiuoli2011}, Cloud BioLinux \cite{Krampis2012} and BioKepler \cite{Altintas2011} 
that are open-source and accessible to any laboratory or research group through the Amazon EC2 computer Cloud 
\cite{awsec2}. In more detail, and as an example of these platforms, the Galaxy bioinformatics workbench, 
includes a range of software, from scripts that extract entries from FASTA files, to tools for processing 
next-generation sequence data. Galaxy a self-contained platform including a web server software stack, a set 
of bioinformatics tools with pre-compiled interfaces. and users can easily add more software packages 
It offers an intuitive, web browser accessible interface for the tools in addition to a drag and drop canvas 
for creating data analysis workflows. The system was specifically designed as a framework for easy deployment 
through a web portal of command-line only software that lack a user interface, by providind a standardized method 
to create intefaces for bioinformatics tools with minimal expertise by editing simple configuration scripts written 
in the eXtensible Markup Language (XML). Furthermore, it allows to extend the computational capacity for genomic data
analysis beyond the capacity of a single compute node through Galaxy-Cloudman \cite{Afgan2010}, the provides Sun 
Grid Engine clusters for parallel computing on Cloud computing platforms such as Amazon EC2.

Another public offering for computing on the Cloud has come through our own work on Cloud Biolinux \cite{Krampis2012},
\cite{cloudbio}, which is a virtual high performance computing server on Amazon EC2 providing the community with 
on-demand bioinformatics computing and a large assortment of pre-configured sequence analysis tools, within 
a high-performance Virtual Machine (VM) server that runs on Cloud computing services such as Amazon EC2. The project is targeted 
to researchers that do not have access to large-scale informatics infrastructure for sequence data analysis, 
allowing them to rent computational capacity from Cloud services. Users can access the tools by starting the 
Cloud BioLinux VM through the Amazon cloud console web page \cite{}, while easily perform large-scale data analysis
as we have demonstrated with the 1000 Human genomes data \cite{},\cite{},\cite{}. The Cloud BioLinux VM is 
open-source, can be downloaded and modified, while advanced users can run the tools on a private installation of 
the Eucalyptus \cite{} or Openstack \cite{} Cloud platforms. A diverse community of researchers from both the US
(Massachusetts General Hospital, Harvard School of Public Health, Emory University) and Europe (National
Environmental Research Center, King's College London, Denmark Technical University, Netherlands Wageninen
University) has been already established around the project \cite{}. Finally, we have recently expanded Cloud
BioLinux by adding support for advanced users through a development framework for building and distributing
customized bioinformatics VMs, enhancing the community aspects of Cloud BioLinux and adding flexibility for 
development of customized, cloud-based bioinformatics data analysis solutions. This framework provides a
software management system that simplifies the process of selecting the bioinformatics tools to be included in
the VM, automates building of a new VM with the specified software, and seamlessly deploys across different
Cloud platforms, while it is freely available from the GitHub code repository \cite{}.  The overall goal is
to offer a platform for maintaining a range of specialized VM setups for serving different computing needs
within the bioinformatics community, and allow researchers to focus on the next challenges of providing data,
documentation, and the development of scalable analysis pipelines. 


While the platforms presented above provide excellent sequence analysis suites that are widely accessible 
on the Cloud, they simply package existing bioinformatics applications within VM servers. This is a great 
solution for smaller laboratories that lack informatics infrastructure for large scale sequencing data analysis 
since they provide pre-configured software and on-demand computing platforms using virtualized infrastructures.  
However, they simply port on the Cloud existing bioinformatics applications that have monolithic designs, that
process data serially, and are not designed to leverage specific characteristics of Cloud computing platforms,
such as highly parallelism and distributed computing.  Therefore, applications cannot scale efficiently as the amount
of data increases, while specialized bioinformatics applications and pipelines that have been implemented at
the big bioinformatics centers to scale parallel for large sequences datasets, are usually coupled with specific 
cluster computing hardware and informatics infrastructures at each center, requiring extended effort or being 
very difficult to refactor the code to run on the Cloud \cite{Wilkening2009}.



\subsection*{Available Options to Buying Software as A Service on the Cloud ?}

During the past year, both public and commercial offerings pre-configured sequence analysis applications on
the Cloud have become available. On the commercial side, DNAnexus (online ref.  3) currently includes tools
for ChiPseq, RNAseq, 3'-end sequencing for expression quantification (3SEQ) and enzyme restriction analysis.
DNAnexus runs on the Amazon Elastic Compute Cloud (EC2, online ref. 4), which provides on-demand virtual
servers with various compute capacities. Another offering is iNquiry, which is a port to the Cloud of the
bioinformatics software suite that used to be bundled together with the computer clusters built by the BioTeam
(online ref. 5), but is no longer maintained. This platform is essentially a web-server for pre-installed
open-source tools such as EMBOSS, HMMER, BLAST and the R statistical package, also on the EC2 Cloud platform.
And many consulting firms such as cycle computing and BioTeam provide custom solutions at high cost.


\subsection*{Can Non-Computationally Savvy Researchers Easily Access the Cloud?} Text for this sub-section.
More results \ldots

VM servers with the pre-configured pipelines and data will be publicly available for download. Researchers
will have the option to execute them on a desktop computer, using virtualization software such as VirtualBox
(http://www.virtualbox.org). VirtualBox is free and can be installed with a single step on Windows, Mac or
Linux desktop computers.  Alternatively, research teams with informatics expertise and access to a local
cluster could choose to download our VM servers and perform large-scale data analysis by running them on a
private Cloud installation, after installing Eucalyptus or OpenStack and converting part of the cluster to a
local Cloud. 

\subsection*{What the Public, Private, Open-Source or Commercial Clouds Available to Biologists Today ?} 

ur system will not be limited only to the Amazon EC2 Cloud platform, since researchers that have access to a
local cluster at their home institution will have the option to download the VM and run, without being
required to perform any software installation.  The only dependency will be a virtualization layer that can
run the VM on the cluster such as for example the OpenStack open-source Cloud (http://www.openstack.org).
OpenStack is available as part of widely used Ubuntu Linux (http://www.ubuntu.com/Cloud) and included by
default on a compute cluster set-up to run this operating system, while it can be easily installed as a
package on clusters running other Linux versions. Alternatively, for researchers that do not have access to
local compute Clouds, OpenStack installations can be accessed through the government-funded Argonne National
Lab Magellan Cloud http://www.alcf.anl.gov/magellan) that provides compute allocations to researchers, in
addition to a number of academic computing centers in both the US and abroad
(http://openstack.org/user-stories/), or commercial Cloud providers such as RackSpace.

\subsection*{Unique characteristics of Cloud Computing versus traditional Bioinformatics infrastructures}

Data storage using a Cloud service has the advantage that large-scale sequencing datasets can be easily
exchanged among collaborators worldwide. Inherent in the design of Amazon S3 (http://aws.amazon.com/s3)
service is replication of data across several physical storage locations for disaster prevention, available
currently on US East and West regions, European Union (Ireland) and Asia Pacific (Singapore). For the data
upload a researcher can choose the closest region for minimizing data transfer latency over the internet.
Following that, the Cloud service automatically replicates the data to different locations as part of the
disaster prevention policy, allowing collaborating researchers to download the data from their closest region.

Currently Amazon S3 has offers a community program (http://aws.amazon.com/datasets) to host a variety of
widely used public datasets at no charge for  researchers.  Bioinformatics-related datasets hosted for free
come from the 1000 human genomes, the complete Genbank, Ensembl and Unigene databases, in addition to the
Ensembl human genome annotation data. public data Researchers can then access, copy, modify and perform
computation on these data volumes directly using pre-configured VMs on Amazon EC2 instances such as JCVI's
Cloud Biolinux, and just pay for the compute and additional storage resources they use. We are in negotiations
with Amazon to get support for hosting the data from this project as part of this program.

customers of our service that would like to perform extensive computational analysis of the assembled genomes
but do not have access to a cluster, we will offer the option to upload the VM portal and assembly data on the
Amazon EC2 Cloud.  Researchers will then be able to rent computational and storage capacity from Amazon, as an
alternative to local informatics infrastructure at their laboratories.  This can work as a better model where
the cost for hardware and data center maintenance cannot be justified for only a few experiments involving
sequencing and genome assembly in smaller research laboratories [56]. The Amazon EC2 Cloud service employs a
charge model similar to traditional utilities such as electricity, and users renting servers are billed based
on the amount of computational capacity consumed on an hourly basis [57]. The Amazon Cloud consists of
thousands of computer servers with petabytes of storage, leveraging economies of scale to get low operational
costs that in turn offers as savings to users. Furthermore, this Cloud service offers data centers in US East
and West regions, European Union and Asia), providing users worldwide the ability to tap into a large pool of
computational resources, outside of institutional, economic or geographic boundaries. For researchers to run
our VM with the pre-installed tools on the Amazon EC2 Cloud, they will only need to follow four simple steps
through their web browser: visit the Amazon Cloud website and create a new account, start the VM execution
“wizard” through the Cloud's control console [58], choose computational capacity for the VM (memory,
processor, cores, storage capacity), and specify username and password credentials for accessing the running
VM. Each running VM receives a unique web address, and by using their web browser to access the address, a
researcher can login to the portal interface with the assembly tools. These four steps are described in detail
in our Cloud BioLinux publication and the project's documentation [59].

Stein LD. (2010) The case for Cloud computing in genome informatics. Genome biology, 2010, 11, 207+ 

Amazon EC2 Cloud pricing : http://aws.amazon.com/ec2/pricing  

Amazon EC2 Cloud console : https://console.aws.amazon.com 

Cloud BioLinux project documentation : http://tinyurl.com/Cloud-docu

Deliver to the genomics community beyond data generation at our center, by capturing and distributing the
bioinformatics “know-how” developed during this project through VMs, which will be made publicly available on
the Cloud and will contain all software required for sequence analysis. One of the major obstacles for using
open-source bioinformatics tools is the need to configure all of the complex dependencies including the type
of operating system, code libraries and computer hardware used where the software was originally developed.
This creates a bottleneck for laboratories lacking the required technical expertise and informatics
infrastructure, which we propose to overcome by distributing VMs with pre-configured bioinformatics tools.
Democratize access to computational resources needed for high throughput sequencing data analysis in order to
stimulate further adoption of the technology in the wider community. Computational analysis can become a major
bottleneck for smaller laboratories and academic institutes transitioning their experimental techniques to
sequencing-based methods (for example RNAseq for gene expression, ChipSeq for protein interactions,
metagenomics for microbial interactions). Our bioinformatics VMs will be readily available for use by research
groups acquiring sequencing capability. They will be distributed through the Amazon EC2 Cloud service, which
provides a publicly accessible, high performance computing platform that can be rented on-demand by
researchers for performing large scale data analysis at low cost, removing the need for implementing
informatics infrastructure at each laboratory.  

For example a large capacity server with 64GB memory and 8 processor cores that would suffice for most types
of bioinformatic analysis, costs 2 US per hour. Given the scale of this Cloud service (data centers in US East
and West regions, European Union and Asia), researchers worldwide will be able to instantiate an unlimited
pool of our VMs for their sequence data analysis needs. Provide a tool for researchers to easily extract or
add value to the data released from the proposed and future sequencing projects. The current approach of data
warehousing in public databases such as the SRA (http://www.ncbi.nlm.nih.gov/sra) offers little value for
scientists that do not have access to computing infrastructures required for large-scale data analysis. We
will demonstrate a new approach for distributing genomic datasets in conjunction with pre-configured
computational capacity, by depositing our data on Amazon S3 and pre-install bioinformatics tools in VMs on
Amazon EC2.  We anticipate that as the genomics community undertakes sequencing projects at similar or smaller
scales, researchers that lack informatics resources and expertise will require a data analysis approach
reducing the steps of: data download, provisioning informatics infrastructure and installing the tools,
perform data analysis and hypothesis testing, integrate with data from previously funded sequencing studies.
By placing our data on a publicly accessible compute platform we will facilitate this, and reduce the number
of required steps to: upload local data and merge with data available on the Cloud, perform data analysis and
discovery using the Cloud-computing infrastructure. Furthermore, renting computational capacity on a per need
basis serves as a better model for smaller research laboratories, instead of investing to computer hardware
and data center infrastructure for which the cost cannot be justified for only a handful of experiments.  


The enabling technology towards is direction is virtualization, which allows data analysis pipelines,
databases, website portals and all software dependencies including entire operating systems, code libraries
and configuration files to be encapsulated in Virtual Machines (VMs). A Virtual Machine is essentially an
emulation of a compute server, albeit a full-featured one, with virtual processors, memory and storage
capacity, in the form of a single binary file. Cloud services such as Amazon web-services
(http://aws.amazon.com) offer high-performance computer hardware with a virtualization layer, upon which a
user executes VMs.  Since the VMs are full-featured compute servers  in the form of a single binary file,
researchers with access to local computing clusters have also the option to download and run the VM servers
and therefore instantiate a local version of the AIP on a private Cloud. For example, a researcher can install
the Eucalyptus or OpenStack Cloud (http://open.eucalyptus.com, http://www.openstack.org) that are an
open-source replicas of the Amazon Cloud on their clusters, while also foregoing the need to perform any
software installation since the VMs contain pre-installed all required dependencies for running the AIP.

Meanwhile advances in cyber-infrastructure and information technology are changing the landscape of biological
computing. Virtualization technologies allow entire compute servers including the operating system replete
with all the necessary software packages for data analysis, to be archived within a Virtual Machine (VM). A VM
is an emulation of a compute server, with virtual processors, memory and storage capacity, in the form of a
single binary file that can be executed independently of the hardware architecture available [3]. Since all
software components and dependencies are encapsulated within the VM, it is possible to distribute
pre-installed, ready-to-execute bioinformatics tools and data analysis pipelines, in the format of a single
binary, down-loadable VM file. This addresses one of the main hurdles to make open-source software with
complex dependencies and installation procedures widely accessible to the research community.  Virtualization
technologies led to the development of Cloud computing services, where remote computer server farms can be
rented on an hourly basis by researchers and used for scalable, on-demand computation. Cloud services offer
high-performance computer hardware with virtualization, upon which a user executes VM servers [3]. Renting
computational capacity from these services has the potential to eliminate many of the upfront capital expenses
of building information technology infrastructure for metagenomic research, and result in transformation of
the analysis and data processing tasks into well defined operational costs. 

Following the concept of Whole System Snapshot Exhange (WSSE, Dudley and Butte 2010) we will create Virtual
Machines (VMs) that contain replicas of the bioinformatics systems used for data analysis in this study. The
VMs will be made publicly available on the Amazon EC2 Cloud computing platform, for the purpose of fulfilling
two goals: first, to allow reproducibility of the bioinformatic analysis performed in this project through
placing the data along with pre-configured software, on a Cloud compute platform accessible by the worldwide
genomics community outside of institutional, economic or national boundaries. Given the complexity of  the
current project, following publication of our results we expect for example researchers to require re-running
part of the bioinformatic analysis for adjusting the algorithmic parameters. Second, as bioinformatics
computing is a key aspect for researchers in the community to extract value from the data released from this
project, while also build additional value on top as similar studies are performed or by integrating with
results from existing studies, placing our data on a publicly accessible compute platform will facilitate this
process. Besides the current study, we believe that this approach will demonstrate a new approach for sharing
research results by distributing  the data together with the computational capacity, which will provide an
option for researchers from smaller institutions without access to extensive infrastructure for their data
analysis needs.

 For researchers that would like to leverage the advantages of a VM with the pre-installed assembly portal for
 working with the completed assemblies but consider the public Cloud as not secure option, we will offer the
 alternative of returning to them by mail an external hard drive with a VM containing with the portal and
 assembly data. Users will then be able to load and execute the VM on a local computer cluster with a
 Eucalyptus/OpenStack Cloud or on a PC using Virtualbox. We are currently offering a similar solution with
 Cloud BioLinux [3], where the project's VM is available for download and execution on a local Cloud or a PC
 from our website [4].  Upon local execution of a VM users will simply need to point their browser to the
 portal's Internet or local IP address [55] assigned automatically by either the Cloud or Virtualbox (Fig.1B).
 The IP address is available through each Cloud platform's or the Virtualbox software administrative
 interface, and we will provide extensive documentation (see subsequent paragraph) on uploading, running and
 accessing a local VM on the different platforms by extending the available Cloud BioLinux project
 documentation.

In our experience, there is no ideal solution for being able to provide long-term support and maintaining
hardware servers that provide public access to bioinformatics software, especially once the funding cycle of a
project runs out.  By using VMs however, we can preserve an exact replica of the AIP portal and databases in
its precise state at the time when the VM was generated. This allows us to archive the system, and users have
the ability to re-instantiate AIP to a fully functional state at any time in the future by installing a local
Cloud or leasing computing time on the Amazon Cloud. Our approach offers a way to keep the system readily
accessible with minimal cost (only hosting the VMs on an FTP site, which can be long-term provided by JCVI’s
IT for free) past the funding cycle, while researchers who receive funding for their own projects can either
allocate on their budget compute costs on the Amazon Cloud or budget in their grant hardware for a local
Cloud.

\subsection*{Which Factors Challenge Adoption of Cloud-Based Solutions for Bioinformatics }

Another important concern for Cloud-based bioinformatic tools is related to the data transfer bottleneck from
the local sequencing machines to the Cloud servers.  According to the data published for the Amazon Cloud
platform (online ref.11), 600GB of data would require approximately one week to upload on to the remote Cloud
servers, when using an average broadband connection of 10Mbps. With a faster T3 connection which is usually
easily obtainable even at small research institutions, within one week 2TB of data can be uploaded or
approximately 600GB in 2 days.  Solutions addressing this issue are available both as software that maximizes
data transfer over the network compared to traditional File Transfer Protocol (FTP), or physical disk drive
import/export services offered by the Cloud provider to its customers. Aspera's server (online ref. 12) has
been recently integrated to NCBI's infrastructure, and researchers can download a free client that allows
increased upload speeds to the Short Read Archive (online ref. 13).  Through the Aspera software, transfer
bandwidth between NCBI and the European Bioinformatics Institute for data sharing in the 1000 Genomes Project,
has been increased from 20Mbps to 1000Mbps (see online ref. 14). 

Finally, the Amazon offers the option for its users to physically ship disk drives to the company's offices
and have the data copied to their servers (online ref. 11). With only 80 import cost for disk drives up to 4TB
of data (4000GB), this is the most efficient method if we take into account the charge by Amazon for 0.10 per
GB of bandwith consumed, which would add up to 60 for a 600GB data upload. In addition to that cost, the
expense for obtaining a high-bandwidth internet connection for the data upload should be taken into account.
We expect the Microsoft Azure Cloud platform to offer a similar service in the near future, given the requests
on the Azure developer forums and the immediate consideration of the matter by Microsoft (online ref. 15).



%%%%%%%%%%%%%%%%%%%%%%
\section*{Conclusions} Text for this section \ldots



%%%%%%%%%%%%%%%%%%
\section*{Methods} \subsection*{Methods sub-heading for this section} Text for this sub-section \ldots

\subsection*{Another methods sub-heading for this section} Text for this sub-section \ldots

\subsection*{Yet another sub-heading for this section} Text for this sub-section \ldots



%%%%%%%%%%%%%%%%%%%%%%%%%%%%%%%%
\section*{Authors contributions} Text for this section \ldots



%%%%%%%%%%%%%%%%%%%%%%%%%%%
\section*{Acknowledgements} \ifthenelse{\boolean{publ}}{\small}{} Text for this section \ldots



%%%%%%%%%%%%%%%%%%%%%%%%%%%%%%%%%%%%%%%%%%%%%%%%%%%%%%%%%%%%%
%%                  The Bibliography                       %%
%%                                                         %%              
%%  Bmc_article.bst  will be used to                       %%
%%  create a .BBL file for submission, which includes      %%
%%  XML structured for BMC.                                %%
%%                                                         %%
%%                                                         %%
%%  Note that the displayed Bibliography will not          %% 
%%  necessarily be rendered by Latex exactly as specified  %%
%%  in the online Instructions for Authors.                %% 
%%                                                         %%
%%%%%%%%%%%%%%%%%%%%%%%%%%%%%%%%%%%%%%%%%%%%%%%%%%%%%%%%%%%%%


{\ifthenelse{\boolean{publ}}{\footnotesize}{\small} \bibliographystyle{bmc_article}  % Style BST file
\bibliography{bmc_article} }     % Bibliography file (usually '*.bib') 

%%%%%%%%%%%

\ifthenelse{\boolean{publ}}{\end{multicols}}{}

%%%%%%%%%%%%%%%%%%%%%%%%%%%%%%%%%%%
%%                               %%
%% Figures                       %%
%%                               %%
%% NB: this is for captions and  %%
%% Titles. All graphics must be  %%
%% submitted separately and NOT  %%
%% included in the Tex document  %%
%%                               %%
%%%%%%%%%%%%%%%%%%%%%%%%%%%%%%%%%%%

%%
%% Do not use \listoffigures as most will included as separate files

\section*{Figures} \subsection*{Figure 1 - Sample figure title} A short description of the figure content
should go here.

\subsection*{Figure 2 - Sample figure title} Figure legend text.





\end{bmcformat} \end{document}







